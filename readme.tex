\documentclass[A4]{ltxdoc}
\usepackage{markdown}   
\usepackage{fontawesome} 
\usepackage{hyperref, xcolor}
\hypersetup{colorlinks=true, linkcolor=myblue, urlcolor=myblue,hyperindex}
\usepackage{gitinfo2}
\usepackage{fancyhdr}
\usepackage{url}

\fancypagestyle{plain}{
    \fancyhf{} % sets both header and footer to nothing
    \renewcommand{\headrulewidth}{0pt}
    \fancyfoot[C]{
        \faGit \textbullet{}
            \textsf{Branch: \gitBranch\,@\,\gitAbbrevHash{}
            Release:\gitReln{} (\gitAuthorDate)}
        } 
}

\pagestyle{plain}
 
\definecolor{myblue}{rgb}{0.22,0.45,0.70}

\begin{document}
\thispagestyle{plain}
\title{gitinfo2 and latexmk integration}

\author{%
  Raphaël P. Barazzutti\thanks{%
    URL: \href{https://github.com/rbarazzutti/gitinfo2-latexmk}{github.com/rbarazzutti/gitinfo2-latexmk};
    Licence: \href{http://latex-project.org/lppl/lppl-1-3c.txt}{LaTeX Project Public License, version 1.3c};
    E-mail: \href{mailto:raphael.barazzutti@unine.ch}{\tt raphael.barazzutti@unine.ch}.}}

\maketitle
\thispagestyle{plain}
\begin{abstract}
    This tiny piece of software allows a painless and seamless integration of \textsf{gitinfo2} with \textsf{latexmk}. Instead of relying on \textsf{git}'s action hooks, this piece of code is loaded
    by \textsf{latexmk} when a build is triggered.
\end{abstract}

\markdownInput{README.md}





\end{document}